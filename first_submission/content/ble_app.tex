\section{BLE library in the Flutter app}
The edge component of the application is an Android/iOS application written using
the Flutter framework\cite{b4} and using the Flutter Blue library\cite{b5} to communicate
with the BLE peripheral.
\subsection{Environment setup}
We program the application using Visual Studio Code with the Flutter extension, and
we use the Android Studio to run the application on an Android device and manage the
Android SDK. \\
To enable the Flutter Blue library, we add the following dependency to the pubspec.yaml file:
\begin{lstlisting}
dependencies:
    [| \ldots |]
    flutter_blue: ^0.8.0
\end{lstlisting}

\subsection{Use of the Flutter Blue library}
The Flutter Blue library gets initialized through the creation of a FlutterBlue object.
This object is used to scan for BLE peripherals, connect to them, and read/write data. \\

This is done by first of all binding some functions to the FlutterBlue objects through
the FlutterBlue.scanResults\#listen method, adding the scanned devices to the containing
widget's state. \\

At that point, the scan can be started with the FlutterBlue\#startScan method, and
a connection with a specific device can be established with the device\#connect
method (using the Device object that we receive when scanning for new connections). \\

As the NOTIFY characteristic is enabled on our device, once a connection is established,
we are able to receive data from the device without having to explicitly ask for it.
This is done through the Characteristic\#setNotifyValue method and the Characteristic\#listen
method, both coming from a Characteristic object that we can extract from the Service
objects returned by the device\#discoverServices method. \\