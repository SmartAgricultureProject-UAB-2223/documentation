\section{Performance requirements}
We can analyze the performance requirements of this system by dividing it in three
parts: requirements for the device, for edge devices (i.e. smartphones), and for
the online server. We will discuss each of these parts separately.

\subsection{Device}
The requirements for the device are simple. No high computational power is required,
since the device will only be used to collect data from sensors and send it to the
server, and the local calculations that are required are very simple, such that
they can be executed by most of the devices that are available on the market. \\
Much more important are size, power consumption, durability and storage requirements
for the device.

\subsubsection{Size}
The device is meant to be inserted into or near the soil, since
the sensors must be inside the pot. This means that, if usage of "standard" 1L plant
pots is supported, the final result can't occupy more space than a circle of 40mm
in diameter (more leeway is given in regards to thickness), plus the sensors.

\subsubsection{Power consumption}
The device is meant to be running constantly, in places where it is not always possible
to have direct access to electricity - plant pots are not usually placed near power
sockets. This means that the device must be able to run on batteries for a long time,
and that the power consumption must be as low as possible. \\
Considering the current state of the art as far as batteries go, it is possible
and thus necessary to develop a device that can run for at least a year on a single
charge. The battery might violate the size requirement, but this could be mitigated
in various ways. \\
The power consumption of the device is also important, since it will be running
constantly. It should not be consuming, on average, on more than a single watt.
This figure takes into account peaks in power consumption, such as when the device
is sending data to the mobile device or extracting data from sensors.

\subsubsection{Durability}
The device must be able to withstand the conditions in which it will be used. This
means that it must be waterproof. More extreme conditions should not be taken into
consideration for now, since any condition that is so extreme as to potentially
damage the device would also be extremely dangerous for the plants, such that there
is no reasonable use case for the device in such conditions.

\subsubsection{Storage}
The device must be able to store the data it collects for a long time, since it will
upload it only when given access to the internet through a mobile device using the
app. Reasonably, the device should at most be able to store data for a few months.
This might mean needing a small amount of flash memory (128MB is, at this point in time,
a reasonable estimate since it's still unknown how much data will be generated for
storage), and possibly implementing some data compression algorithm to reduce the
amount of data that needs to be stored.

\subsection{Edge}
The software requirements are also quite simple. The software that runs on the device
must be able to collect data from the sensors and send it to the server through a
mobile application. The application itself should run on modern smartphones using
any of the major operating systems (Android, iOS). The application would store the
data, and execute calculations on it locally. It must also be able to upload this
data to a server for backups and social sharing.\\
The need for direct communication between a Smartphone app and the device greatly
restricts the possible solutions as far as data transfer goes. Any protocol used
must be supported by devices running any modern mobile operating system. Additionally,
the limited throughput of some of the protocols (such as Bluetooth) means that care
should be put into both reducing the amount of transfered data, and taking care of
transfer errors and interruptions that would be made much more common the longer
the transfer takes.

\subsection{Server}
The server itself only needs to be able to store the data and implement the most
basic of authentication mechanisms. There is the possibility of having it execute
the calculations on the data, instead of having the mobile device upload the results,
but at the current time this does not seem necessary. This means that the server's
performance requirements are quite low, especially during the launch phase. Additionally,
modern web cloud technology allows for dynamic scaling of server resources, such
that the current estimate only needs to take into account the initial user base
of the application.